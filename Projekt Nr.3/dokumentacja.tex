\documentclass[a4paper, 11pt]{article}
\author{Kajetan Kaczmarek}
\usepackage{amsmath}
\usepackage{tabularx}
\usepackage{graphicx}
\usepackage{listings}
\usepackage[T1]{fontenc}
\usepackage[utf8]{inputenc}
\usepackage[polish]{babel}
\usepackage{color} %red, green, blue, yellow, cyan, magenta, black, white
\definecolor{mygreen}{RGB}{28,172,0} % color values Red, Green, Blue
\definecolor{mylilas}{RGB}{170,55,241}


\lstset{language=Matlab,%
    %basicstyle=\color{red},
    basicstyle=\tiny,
    breaklines=true,%
    morekeywords={matlab2tikz},
    keywordstyle=\color{blue},%
    morekeywords=[2]{1}, keywordstyle=[2]{\color{black}},
    identifierstyle=\color{black},%
    stringstyle=\color{mylilas},
    commentstyle=\color{mygreen},%
    showstringspaces=false,%without this there will be a symbol in the places where there is a space
    numbers=left,%
    numberstyle={\tiny \color{black}},% size of the numbers
    numbersep=9pt, % this defines how far the numbers are from the text
    emph=[1]{for,end,break},emphstyle=[1]\color{red}, %some words to emphasise
    %emph=[2]{word1,word2}, emphstyle=[2]{style},    
}



\begin{document}
\title{Sprawozdanie MNUM \\* Projekt nr.3 \\* 
Zadanie 3.32 \\*}
\maketitle

\begin{enumerate}

\item Opis zastosowanych algorytmów : 
\begin{enumerate}
\item W pierwszym zadaniu, tj. znalezienie zer dla funkcji \[f(x) = 0.5xcos(x) - ln(x) \] użyłem dwóch metod. Założeniami dla obywdu metod była a) ciągłość, co jest oczywiste dla ww. funkcji, oraz b) różne znaki na krańcach przedziału, do czego odnoszę się poniżej. Zastosowane metody : 
\begin{itemize} \, \item Metoda bisekcji \\* \, W metodzie bisekcji na początek liczony jest punkt wypadający pomiędzy podanymi wejściowymi punktami, tj. \( x = \dfrac{a+b}{2}\) dla p. wejściowych a i b. Następnie sprawdzamy czy punkt ten jest naszym zerem z podaną dokładnością eps, czyli czy \( |f(x)| < eps \). Jeśli tak jest kończymy wykonywanie algorytmu, jeśli nie to sprawdzamy warunek \( f(a) f(b) < 0 \) i w zależności od wyniku zastępujemy lewy lub prawy koniec przedziału w którym szukamy wyliczonym x, tak, aby krańce przedziału nadal miały przeciwne znaki. Alternatywnym warunkiem wyjściowym z pętli jest \( |a-b|<eps \), czyli zbliżenie się do siebie punktów a i b tak, że dalsze obliczenia są niemożliwe.
\item Metoda Siecznych \\* Metoda siecznych jest podobna do metody bisekcji - szukamy zer przez zawężanie zakresu poszukiwań, warunki końcowe są więc takie same.Różny jest jednak algorytm wyznaczania kolejnego punktu : tutaj kolejne punkty wyznaczamy ze wzoru \[ x_i = x_{i-1} - f(x_{i-1})\dfrac{x_{i-1} - x_{i-2}}{f(x_{i-1}) - f(x_{i-2})} \]
Tak że łączenie kolejnych punktów daje nam sieczne naszej funkcji f(x) i przybliża jej zera. W ten sposób kolejne punkty są łączone prostą a jej przecięcie z osią x wyznacza kolejny punkt
\item Uwaga techniczna - założeniami obydwu metod są różne znaki funkcji na krańcach przedziału. Jako że warunek ten nie jest spełniony dla zadanego przedziału w mojej funkcji, a do tego ww. metody znajdują tylko jedno zero, podzieliłem zadany przedział [2,11] na dwa mniejsze , tj. [2,7] i [7,11] tak aby w każdym znajdowało się jedno zero, i aby spełniały one założenia metod.
\end{itemize}
\item W drugim zadaniu, tj. znalezienie zer wielomianu \( f(x) = 2x^4 + 5x^3 - 2x^2 +3x^3 +7 \) zastosowałem metody : 
\begin{itemize}
 \item Metoda Newtona \\* Metoda Newtona, zwana inaczej metodą stycznych , opiera się na wykorzystaniu iteracjnego wzoru \[ x_i = x_{i-1} - \dfrac{f(x)}{f(x)'} \] , który wynika z rozwinięcia funkcji w szereg Taylora, czyli  \[f(x) =~f(x_n) +f'(x_n)(x-x_n) \] . Metoda Newtona jest lokalnie zbieżna, dla punktów zbytnio oddalonych od p. zerowego może nie dawać rezultatów.
\item Metoda Mullera MM2 \\* Ogólną ideą metod Mullera jest przybliżanie lokalnie funkcji w okolicy zera przez funkcję kwadratową. Jest pewnym uogólnieniem metody siecznych, z dodaną liniową interpolacją pomiędzy kolejnymi punktami. Werjsa MM2 algorytmu, użyta w moim rozwiązaniu, zakłada użycie wartości wielomianu i jego pierwszej oraz drugiej pochodnej w danym punkcie.Kolejne punkty wyliczamy z wzoru \[ x_{k+1} = x_k + z_{min} , \] gdzie \(z_{min}\) jest jedną z wartości  \\* \[ z_{+/-} = \dfrac{-2f(x_k)}{f'(x_k) +/- \sqrt{(f'(x_k))^2 - 2f(x_k)f''(x_k)}} \] w zależności od modułu mianownika - wybieramy tę z większym modułem
\end{itemize}
\end{enumerate}

\item Kod moich programów 
\begin{itemize}
\item Funkcja main dla pierwszego zadania \\*
\lstinputlisting{P1_Main.m}
\item Pomocnicza funkcja licząca wartości naszej funkcji podanej dla zadania \\*
 \lstinputlisting{fzad.m}
 \item Funkcja licząca zera funkcji metodą bisekcji
 \lstinputlisting{bisect.m}
 \item Funkcja licząca zera funkcji metodą siecznych
 \lstinputlisting{secants.m}
  \item Funkcja main dla drugiego zadania
 \lstinputlisting{P2_Main.m}
   \item Funkcja licząca zera funkcji metodą Newtona
 \lstinputlisting{newton.m}
   \item Funkcja licząca zera funkcji metodą Mullera
 \lstinputlisting{muller.m}
\end{itemize}
\item
Wyniki : 
\begin{itemize}
\item  Dla zadania pierwszego obydwie metody zwróciły zbliżone wyniki, tj.\\*
\begin{center}

	\begin{tabular}{ l*{2}{c}r}
  \hline	
	Metoda & Zero nr.1 & Zero nr. 2 \\ \hline
    Metoda Bisekcji & 7.27703857421875	& 5.38775634765625	\\
    Metoda Siecznych & 7.27702154631274	& 5.38773923503257 \\
  \hline
  \end{tabular}

\end{center}
\includegraphics[width=\textwidth, height=\textheight, keepaspectratio]{Zad_1.jpg}
\item Wyniki otrzymane w drugim zadaniu \\* 
\begin{table}    
\begin{center}                                                                          
\centering                                                                                 
\begin{tabular}{|c|c|c|c|c|c|}                                                             
\hline                                                                                     
Iteracja & -5.0000 & -4.5000 & -4.0000 & -3.5000 & -3.0000 \\                                
\hline                                                                                     
1.0000 & -3.6502+0.8489i & -3.3350+0.6876i & -3.0269+0.4954i & -2.7311+0.1819i & -2.8808 \\
\hline                                                                                     
2.0000 & -3.1665+0.0394i & -3.0097+0.0430i & -2.9051+0.0316i & -2.8777-0.0021i & -2.8814 \\
\hline                                                                                     
3.0000 & -2.8724-0.0044i & -2.8809-0.0008i & -2.8814-0.0000i & -2.8814+0.0000i & 0.0000 \\ 
\hline                                                                                     
4.0000 & -2.8814+0.0000i & -2.8814+0.0000i & -2.8814+0.0000i & 0.0000 & 0.0000 \\          
\hline                                                                                     
5.0000 & -2.8814+0.0000i & 0.0000 & 0.0000 & 0.0000 & 0.0000 \\                            
\hline                                                                                     
6.0000 & 0.0000 & 0.0000 & 0.0000 & 0.0000 & 0.0000 \\                                     
\hline                                                                                     
7.0000 & 0.0000 & 0.0000 & 0.0000 & 0.0000 & 0.0000 \\                                     
\hline                                                                                     
8.0000 & 0.0000 & 0.0000 & 0.0000 & 0.0000 & 0.0000 \\                                     
\hline                                                                                     
9.0000 & 0.0000 & 0.0000 & 0.0000 & 0.0000 & 0.0000 \\                                     
\hline                                                                                     
\end{tabular}                          
\caption{Wyniki dla algorytmu Mullera cz.1}                                                                                                                                                                                          
\label{table:Wyniki dla algorytmu Mullera}                                                                                                                                                                                        
\end{center}
\end{table}                                                                                                                                                                                                       


\begin{table}                                                        
\centering                                                           
\begin{tabular}{|c|c|c|c|c|c|c|}                                     
\hline                                                               
Iteracja & -2.5000 & -2.0000 & -1.5000 & -1.0000 & -0.5000 & 0.0000 \\ 
\hline                                                               
1.0000 & -2.9004 & -1.2261 & -0.9863 & -0.9266 & -0.9274 & -1.2656 \\
\hline                                                               
2.0000 & -2.8814 & -0.9345 & -0.9266 & -0.9266 & -0.9266 & -0.9384 \\
\hline                                                               
3.0000 & -2.8814 & -0.9266 & -0.9266 & 0.0000 & 0.0000 & -0.9266 \\  
\hline                                                               
4.0000 & 0.0000 & 0.0000 & 0.0000 & 0.0000 & 0.0000 & 0.0000 \\      
\hline                                                               
5.0000 & 0.0000 & 0.0000 & 0.0000 & 0.0000 & 0.0000 & 0.0000 \\      
\hline                                                               
6.0000 & 0.0000 & 0.0000 & 0.0000 & 0.0000 & 0.0000 & 0.0000 \\      
\hline                                                               
7.0000 & 0.0000 & 0.0000 & 0.0000 & 0.0000 & 0.0000 & 0.0000 \\      
\hline                                                               
8.0000 & 0.0000 & 0.0000 & 0.0000 & 0.0000 & 0.0000 & 0.0000 \\      
\hline                                                               
9.0000 & 0.0000 & 0.0000 & 0.0000 & 0.0000 & 0.0000 & 0.0000 \\     
\hline                                                               
\end{tabular}     
\caption{Wyniki dla algorytmu Mullera cz.2}                                                                                                                                                                                          
\label{table:Wyniki dla algorytmu Mullera}                                                                                                                                                                                        
\end{table}                                                                                                                                                                                                       

\begin{table}                                                                                 
\centering                                                                                    
\begin{tabular}{|c|c|c|c|c|c|}                                                                
\hline                                                                                        
Iteracja & 0.5000 & 1.0000 & 1.5000 & 2.0000 & 2.5000 \\                                        
\hline                                                                                        
1.0000 & 0.1618+0.9566i & 0.5600+0.6375i & 0.8921+0.5884i & 1.2171+0.6326i & 1.5419+0.7170i \\
\hline                                                                                        
2.0000 & 0.6371+0.8950i & 0.6467+0.9535i & 0.6543+0.9070i & 0.7428+0.9098i & 0.8652+0.9797i \\
\hline                                                                                        
3.0000 & 0.6540+0.9399i & 0.6540+0.9398i & 0.6540+0.9398i & 0.6544+0.9400i & 0.6525+0.9451i \\
\hline                                                                                        
4.0000 & 0.6540+0.9398i & 0.6540+0.9398i & 0.6540+0.9398i & 0.6540+0.9398i & 0.6540+0.9398i \\
\hline                                                                                        
5.0000 & 0.0000 & 0.0000 & 0.0000 & 0.0000 & 0.0000 \\                                        
\hline                                                                                        
6.0000 & 0.0000 & 0.0000 & 0.0000 & 0.0000 & 0.0000 \\                                        
\hline                                                                                        
7.0000 & 0.0000 & 0.0000 & 0.0000 & 0.0000 & 0.0000 \\                                        
\hline                                                                                        
8.0000 & 0.0000 & 0.0000 & 0.0000 & 0.0000 & 0.0000 \\                                        
\hline                                                                                        
9.0000 & 0.0000 & 0.0000 & 0.0000 & 0.0000 & 0.0000 \\                                        
\hline                                                                                        
\end{tabular}  
\caption{Wyniki dla algorytmu Mullera cz.3}                                                                                                                                                                                          
\label{table:Wyniki dla algorytmu Mullera}                                                                                                                                                                                        
\end{table}                                                                                                                                                                                                       
     


\begin{table}                                                                  
\centering                                                                     
\begin{tabular}{|c|c|c|c|c|c|c|c|}                                             
\hline                                                                         
Iteracja & -5.0000 & -4.5000 & -4.0000 & -3.5000 & -3.0000 & -2.5000 & -2.0000 \\
\hline                                                                         
1.0000 & -4.0581 & -3.7146 & -3.3874 & -3.0940 & -2.8939 & -3.2123 & 0.1429 \\ 
\hline                                                                         
2.0000 & -3.4242 & -3.2137 & -3.0368 & -2.9176 & -2.8816 & -2.9590 & -2.5414 \\
\hline                                                                         
3.0000 & -3.0550 & -2.9596 & -2.9020 & -2.8827 & -2.8814 & -2.8871 & -3.1143 \\
\hline                                                                         
4.0000 & -2.9066 & -2.8871 & -2.8818 & -2.8814 & 0.0000 & -2.8814 & -2.9238 \\ 
\hline                                                                         
5.0000 & -2.8821 & -2.8815 & -2.8814 & -2.8814 & 0.0000 & -2.8814 & -2.8832 \\ 
\hline                                                                         
6.0000 & -2.8814 & -2.8814 & 0.0000 & 0.0000 & 0.0000 & 0.0000 & -2.8814 \\    
\hline                                                                         
7.0000 & -2.8814 & 0.0000 & 0.0000 & 0.0000 & 0.0000 & 0.0000 & -2.8814 \\     
\hline                                                                         
8.0000 & 0.0000 & 0.0000 & 0.0000 & 0.0000 & 0.0000 & 0.0000 & 0.0000 \\       
\hline                                                                         
9.0000 & 0.0000 & 0.0000 & 0.0000 & 0.0000 & 0.0000 & 0.0000 & 0.0000 \\       
\hline                                                                         
\end{tabular}       
\caption{Wyniki dla algorytmu Newtona cz.1}                                                                                                                                                                                          
\label{table:Wyniki dla algorytmu Newtona}                                                                                                                                                                                        
\end{table}                                                                                                                                                                                                      \begin{table}                                                                            
\centering                                                                               
\begin{tabular}{|c|c|c|c|c|c|c|c|c|}                                                     
\hline                                                                                   
Iteracja & -2.5000 & -2.0000 & -1.5000 & -1.0000 & -0.5000 & 0.0000 & 0.5000 & 1.0000 \\   
\hline                                                                                   
1.0000 & -3.2123 & 0.1429 & -0.9444 & -0.9286 & -1.0806 & -2.3333 & -1.0217 & 0.3182 \\  
\hline                                                                                   
2.0000 & -2.9590 & -2.5414 & -0.9267 & -0.9266 & -0.9341 & -4.3155 & -0.9298 & -1.9463 \\
\hline                                                                                   
3.0000 & -2.8871 & -3.1143 & -0.9266 & -0.9266 & -0.9266 & -3.5914 & -0.9266 & -0.2560 \\
\hline                                                                                   
4.0000 & -2.8814 & -2.9238 & 0.0000 & 0.0000 & -0.9266 & -3.1435 & -0.9266 & -1.4925 \\  
\hline                                                                                   
5.0000 & -2.8814 & -2.8832 & 0.0000 & 0.0000 & 0.0000 & -2.9336 & 0.0000 & -0.9457 \\    
\hline                                                                                   
6.0000 & 0.0000 & -2.8814 & 0.0000 & 0.0000 & 0.0000 & -2.8840 & 0.0000 & -0.9267 \\     
\hline                                                                                   
7.0000 & 0.0000 & -2.8814 & 0.0000 & 0.0000 & 0.0000 & -2.8814 & 0.0000 & -0.9266 \\     
\hline                                                                                   
8.0000 & 0.0000 & 0.0000 & 0.0000 & 0.0000 & 0.0000 & -2.8814 & 0.0000 & 0.0000 \\       
\hline                                                                                   
9.0000 & 0.0000 & 0.0000 & 0.0000 & 0.0000 & 0.0000 & 0.0000 & 0.0000 & 0.0000 \\       
\hline                                                                                   
\end{tabular}    
 \caption{Wyniki dla algorytmu Newtona cz.2}                                                                                                                                                                                          
\label{table:Wyniki dla algorytmu Newtona}                                                                                                                                                                                        
\end{table}                                                                                                                                                                                                       
       
\begin{table}                                    
\centering                                       
\begin{tabular}{|c|c|c|c|c|}                     
\hline                                           
Iteracja & 1.5000 & 2.0000 & 2.5000 & 3.0000 \\    
\hline                                           
1.0000 & 0.9113 & 1.3529 & 1.7527 & 2.1374 \\    
\hline                                           
2.0000 & 0.1704 & 0.7626 & 1.1430 & 1.4652 \\    
\hline                                           
3.0000 & -2.5071 & -0.1390 & 0.5192 & 0.8773 \\  
\hline                                           
4.0000 & -3.1931 & -1.8468 & -0.9372 & 0.1075 \\ 
\hline                                           
5.0000 & -2.9516 & -0.6287 & -0.9266 & -2.5460 \\
\hline                                           
6.0000 & -2.8861 & -0.9876 & -0.9266 & -3.1051 \\
\hline                                           
7.0000 & -2.8814 & -0.9280 & 0.0000 & -2.9210 \\ 
\hline                                           
8.0000 & -2.8814 & -0.9266 & 0.0000 & -2.8830 \\ 
\hline                                           
9.0000 & 0.0000 & 0.0000 & 0.0000 & 0.0000 \\    
\hline                                           
\end{tabular}  
 \caption{Wyniki dla algorytmu Newtona cz.3}                                                                                                                                                                                          
\label{table:Wyniki dla algorytmu Newtona}                                                                                                                                                                                        
\end{table}                                                                                                                                                                                                       
          
\end{itemize}
\item Wnioski : 
\end{enumerate}
\end{document}

