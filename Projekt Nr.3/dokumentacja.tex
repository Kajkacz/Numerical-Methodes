\documentclass[a4paper, 11pt]{article}
\author{Kajetan Kaczmarek}
\usepackage{amsmath}

\usepackage{tabularx}
\usepackage{graphicx}
\usepackage{listings}
\usepackage[T1]{fontenc}
\usepackage[utf8]{inputenc}
\usepackage[polish]{babel}
\usepackage{color} %red, green, blue, yellow, cyan, magenta, black, white
\definecolor{mygreen}{RGB}{28,172,0} % color values Red, Green, Blue
\definecolor{mylilas}{RGB}{170,55,241}


\lstset{language=Matlab,%
    %basicstyle=\color{red},
    basicstyle=\tiny,
    breaklines=true,%
    morekeywords={matlab2tikz},
    keywordstyle=\color{blue},%
    morekeywords=[2]{1}, keywordstyle=[2]{\color{black}},
    identifierstyle=\color{black},%
    stringstyle=\color{mylilas},
    commentstyle=\color{mygreen},%
    showstringspaces=false,%without this there will be a symbol in the places where there is a space
    numbers=left,%
    numberstyle={\tiny \color{black}},% size of the numbers
    numbersep=9pt, % this defines how far the numbers are from the text
    emph=[1]{for,end,break},emphstyle=[1]\color{red}, %some words to emphasise
    %emph=[2]{word1,word2}, emphstyle=[2]{style},    
}



\begin{document}
\title{Sprawozdanie MNUM \\* Projekt nr.3 \\* 
Zadanie 3.32 \\*}
\maketitle

\begin{enumerate}

\item Opis zastosowanych algorytmów : 
\begin{enumerate}
\item W pierwszym zadaniu, tj. znalezienie zer dla funkcji \[f(x) = 0.5xcos(x) - ln(x) \] użyłem dwóch metod. Założeniami dla obywdu metod była a) ciągłość, co jest oczywiste dla ww. funkcji, oraz b) różne znaki na krańcach przedziału, do czego odnoszę się poniżej. Zastosowane metody : 
\begin{itemize} \, \item Metoda bisekcji \\* \, W metodzie bisekcji na początek liczony jest punkt wypadający pomiędzy podanymi wejściowymi punktami, tj. \( x = \dfrac{a+b}{2}\) dla p. wejściowych a i b. Następnie sprawdzamy czy punkt ten jest naszym zerem z podaną dokładnością eps, czyli czy \( |f(x)| < eps \). Jeśli tak jest kończymy wykonywanie algorytmu, jeśli nie to sprawdzamy warunek \( f(a) f(b) < \) i w zależności od wyniku zastępujemy lewy lub prawy koniec przedziału w którym szukamy wyliczonym x, tak, aby krańce przedziału nadal miały przeciwne znaki. Alternatywnym warunkiem wyjściowym z pętli jest \( |a-b|<eps \), czyli zbliżenie się do siebie punktów a i b tak, że dalsze obliczenia są niemożliwe.
\item Metoda Siecznych \\* Metoda siecznych jest podobna do metody bisekcji - szukamy zer przez zawężanie zakresu poszukiwań, warunki końcowe są więc takie same.Różny jest jednak algorytm wyznaczania kolejnego punktu : tutaj kolejne punkty wyznaczamy ze wzoru \[ x_i = x_{i-1} - f(x_{i-1})\dfrac{x_{i-1} - x_{i-2}}{f(x_{i-1}) - f(x_{i-2})} \]
Tak że łączenie kolejnych punktów daje nam sieczne naszej funkcji f(x) i przybliża jej zera. W ten sposób kolejne punkty są łączone prostą a jej przecięcie z osią x wyznacza kolejny punkt
\item Uwaga techniczna - założeniami obydwu metod są różne znaki funkcji na krańcach przedziału. Jako że warunek ten nie jest spełniony dla zadanego przedziału w mojej funkcji, a do tego ww. metody znajdują tylko jedno zero, podzieliłem zadany przedział [2,11] na dwa mniejsze , tj. [2,7] i [7,11] tak aby w każdym znajdowało się jedno zero, i aby spełniały one założenia metod.
\end{itemize}
\item W drugim zadaniu, tj. znalezienie zer wielomianu \( f(x) = 2x^4 + 5x^3 - 2x^2 +3x^3 +7 \) zastosowałem metody : 
\begin{itemize}
 \item Metoda Newtona \\* Metoda Newtona, zwana inaczej metodą stycznych , opiera się na wykorzystaniu iteracjnego wzoru \[ x_i = x_{i-1} - \dfrac{f(x)}{f(x)'} \] , który wynika z rozwinięcia funkcji w szereg Taylora, czyli  \[f(x) =~f(x_n) +f'(x_n)(x-x_n) \] . Metoda Newtona jest lokalnie zbieżna, dla punktów zbytnio oddalonych od p. zerowego może nie dawać rezultatów.
\item Metoda Mullera MM2 \\* Ogólną ideą metod Mullera jest przybliżanie lokalnie funkcji w okolicy zera przez funkcję kwadratową. Jest pewnym uogólnieniem metody siecznych, z dodaną liniową interpolacją pomiędzy kolejnymi punktami. Werjsa MM2 algorytmu, użyta w moim rozwiązaniu, zakłada użycie wartości wielomianu i jego pierwszej oraz drugiej pochodnej w danym punkcie.Kolejne punkty wyliczamy z wzoru \[ x_{k+1} = x_k + z_{min} , \] gdzie \(z_{min}\) jest jedną z wartości  \\* \[ z_{+/-} = \dfrac{-2f(x_k)}{f'(x_k) +/- \sqrt{(f'(x_k))^2 - 2f(x_k)f''(x_k)}} \] w zależności od modułu mianownika - wybieramy tę z większym modułem
\end{itemize}
\end{enumerate}

\item Kod moich programów 
\begin{itemize}
\item Funkcja main dla pierwszego zadania \\*
\lstinputlisting{P1_Main.m}
\item Pomocnicza funkcja licząca wartości naszej funkcji podanej dla zadania \\*
 \lstinputlisting{fzad.m}
 \item Funkcja licząca zera funkcji metodą bisekcji
 \lstinputlisting{bisect.m}
 \item Funkcja licząca zera funkcji metodą siecznych
 \lstinputlisting{secants.m}
  \item Funkcja main dla drugiego zadania
 \lstinputlisting{P2_Main.m}
   \item Funkcja licząca zera funkcji metodą Newtona
 \lstinputlisting{newton.m}
   \item Funkcja licząca zera funkcji metodą Mullera
 \lstinputlisting{muller.m}
    \item Funkcja licząca zera funkcji metodą Newtona z uwzględnieniem deflacji
 \lstinputlisting{newtonDef.m}
   \item Funkcja licząca zera funkcji metodą Mullera z uwzględnieniem deflacji
 \lstinputlisting{mullerDef.m}
\end{itemize}
\item
Wyniki : 
\begin{itemize}
\item  Dla zadania pierwszego obydwie metody zwróciły zbliżone wyniki, tj.\\*
\begin{center}

	\begin{tabular}{ l*{2}{c}r}
  \hline	
	Metoda & Zero nr.1 & Zero nr. 2 \\ \hline
    Metoda Bisekcji & 7.27703857421875	& 5.38775634765625	\\
    Metoda Siecznych & 7.27702154631274	& 5.38773923503257 \\
  \hline
  \end{tabular}
\begin{table}[p]                                                                  
\centering                                                                       
\begin{tabular}{|c|c|c|c|c|c|c|c|c|}                                             
\hline                                                                           
                                                                        
Iteracja &\multicolumn{2}{c|}{Metoda Bisekcji} & \multicolumn{2}{c|}{Metoda Bisekcji}  & \multicolumn{2}{c|}{Metoda Siecznych}  & \multicolumn{2}{c|}{Metoda Siecznych}  \\     
                                                               
 &\multicolumn{2}{c|}{1 przedział}  & \multicolumn{2}{c|}{2 przedział}   & \multicolumn{2}{c|}{1 przedział}  & \multicolumn{2}{c|}{2 przedział}  \\     
\hline   
Dane & X & Y & X & Y & X & Y  & X & Y\\ \hline                                                                                                                                                  
1 & 9 & -6.2973 & 4.5000 & -1.9784 & 7.9037 & -2.2637 & 5.0779 & -0.7175 \\
\hline                                                                          
2 & 8 & -2.6614 & 5.7500 & 0.7267 & 7.2118 & 0.1841 & 6.0558 & 1.1489 \\   
\hline                                                                          
3 & 7.5000 & -0.7150 & 5.1250 & -0.6066 & 7.2638 & 0.0383 & 5.4538 & 0.1452 \\  
\hline                                                                          
4 & 7.2500 & 0.0777 & 5.4375 & 0.1098 & 7.2775 & -0.0013 & 5.3667 & -0.0471 \\  
\hline                                                                          
5 & 7.3750 & -0.2986 & 5.2812 & -0.2417 & &  & 5.3880 & 0.0006 \\     
\hline                                                                          
6 & 7.3125 & -0.1051 & 5.3594 & -0.0636 & &  &  &  \\        
\hline                                                                          
7 & 7.2812 & -0.0123 & 5.3984 & 0.0238 &  &  & &  \\         
\hline                                                                          
8 & 7.2656 & 0.0330 & 5.3789 & -0.0197 & &  &  &  \\         
\hline                                                                          
9 & 7.2734 & 0.0104 & 5.3887 & 0.0021 & &  &  &  \\          
\hline                                                                          
10 & 7.2773 & -0.0009 & 5.3838 & -0.0088 &  &  &  & \\ 
\hline                                                                          
\end{tabular}                                           
\caption{Wyniki metody Bisekcji i Siecznych}                                     
\label{table:Wyniki metody Bisekcji i Siecznych}                                 
\end{table}     
\end{center}
\includegraphics[width=\textwidth, height=\textheight, keepaspectratio]{Zad_1.jpg}

\begin{table}[h]                                                    
\centering                                                       
\begin{tabular}{|c|c|c|c|c|c|c|}                                 
\hline                                                           
\( X_0 \) & \multicolumn{2}{c|}{-6 } & \multicolumn{2}{c|}{-5.500 } & \multicolumn{2}{c|}{-5 } \\     
\hline                                                           
Dane & X & Y & X & Y & X & Y \\                                 
\hline                                                           
1  & -4.769 & 439.495 & -4.411 & 282.809 & -4.058 & 170.154 \\
\hline                                                           
2  & -3.898 & 130.539 & -3.655 & 82.062 & -3.424 & 47.484 \\  
\hline                                                           
3  & -3.324 & 35.443 & -3.179 & 20.895 & -3.055 & 10.815 \\   
\hline                                                           
4  & -3.007 & 7.449 & -2.946 & 3.630 & -2.907 & 1.351 \\      
\hline                                                           
5  & -2.895 & 0.737 & -2.885 & 0.211 & -2.882 & 0.033 \\      
\hline                                                           
6  & -2.882 & 0.010 & -2.881 & 0.001 & -2.881 & 0 \\      
\hline                                                           
7  & -2.881 & 0  & -2.881 & 0 & -2.881 & 0 \\      
\hline                                                           
8  &  &  &  &  &  &  \\         
\hline                                                           
9  &  &  &  &  &  &  \\         
\hline                                                           
\end{tabular}                                                    
\caption{Wyniki metody Newtona cz. 1}                            
\label{table:Wyniki metody Newtona cz. 1}                        
\end{table}                                                      
\begin{table}[p]                                                 
\centering                                                    
\begin{tabular}{|c|c|c|c|c|c|c|}                              
\hline                                                        

\( X_0 \) & \multicolumn{2}{c|}{-4.500} & \multicolumn{2}{c|}{-4 } & \multicolumn{2}{c|}{-3.500} \\  
\hline                                                        
Dane & X & Y & X & Y & X & Y \\                              
\hline                                                        
1  & -3.715 & 92.767 & -3.387 & 42.867 & -3.094 & 13.761 \\
\hline                                                        
2  & -3.214 & 24.080 & -3.037 & 9.510 & -2.918 & 1.963 \\  
\hline                                                        
3  & -2.960 & 4.430 & -2.902 & 1.100 & -2.883 & 0.068 \\   
\hline                                                        
4  & -2.887 & 0.301 & -2.882 & 0.022 & -2.881 &  0\\   
\hline                                                        
5  & -2.881 & 0.002 & -2.881 & 0 & -2.881 &  0\\   
\hline                                                        
6  & -2.881 & 0 &  &  &  &  \\     
\hline                                                        
7  &  &  &  &  &  &  \\      
\hline                                                        
8  &  &  &  &  &  &  \\      
\hline                                                        
9  &  &  &  &  &  &  \\      
\hline                                                        
\end{tabular}                                                 
\caption{Wyniki metody Newtona cz. 2}                         
\label{table:Wyniki metody Newtona cz. 2}                     
\end{table}                                                   
\begin{table}[p]                                                
\centering                                                   
\begin{tabular}{|c|c|c|c|c|c|c|}                             
\hline                                                       
\( X_0 \) & \multicolumn{2}{c|}{-3  }&\multicolumn{2}{c|}{ -2.500}& \multicolumn{2}{c|}{-2  } \\ 
\hline                                                       
Dane & X & Y & X & Y & X & Y \\                             
\hline                                                       
1  & -2.894 & 0.664 & -3.212 & 23.950 & 0.143 & 7.403 \\  
\hline                                                       
2  & -2.882 & 0.008 & -2.959 & 4.397 & -2.541 & -12.183 \\
\hline                                                       
3  & -2.881 & 0 & -2.887 & 0.297 & -3.114 & 15.368 \\ 
\hline                                                       
4  &  &  & -2.881 & 0.002 & -2.924 & 2.319 \\   
\hline                                                       
5  &  &  & -2.881 & 0 & -2.883 & 0.093 \\   
\hline                                                       
6  &  &  &  &  & -2.881 &  0\\    
\hline                                                       
7  &  &  &  &  & -2.881 & 0 \\    
\hline                                                       
8  &  &  &  &  &  &  \\     
\hline                                                       
9  &  &  &  &  &  &  \\     
\hline                                                       
\end{tabular}                                                
\caption{Wyniki metody Newtona cz. 3}                        
\label{table:Wyniki metody Newtona cz. 3}                    
\end{table}                                                  
\begin{table}[p]                                                 
\centering                                                    
\begin{tabular}{|c|c|c|c|c|c|c|}                              
\hline                                                        
\( X_0 \)  & \multicolumn{2}{c|}{-1.500}& \multicolumn{2}{c|}{-1  }&\multicolumn{2}{c|}{-0.500 }\\  
\hline                                                        
Dane & X & Y & X & Y & X & Y \\                              
\hline                                                        
1  & -0.944 & -0.238 & -0.929 & -0.027 & -1.081 & -2.160 \\
\hline                                                        
2  & -0.927 & -0.002 & -0.927 & 0 & -0.934 & -0.101 \\
\hline                                                        
3  & -0.927 & 0 & -0.927 & 0 & -0.927 & 0 \\
\hline                                                        
4  &  &  &  &  & -0.927 &  0\\    
\hline                                                        
5  &  &  &  &  &  &  \\      
\hline                                                        
6  &  &  &  &  &  &  \\      
\hline                                                        
7  &  &  &  &  &  &  \\      
\hline                                                        
8  &  &  &  &  &  &  \\      
\hline                                                        
9  &  &  &  &  &  &  \\      
\hline                                                        
\end{tabular}                                                 
\caption{Wyniki metody Newtona cz. 4}                         
\label{table:Wyniki metody Newtona cz. 4}                     
\end{table}                                                   
\begin{table}[p]                                                   
\centering                                                      
\begin{tabular}{|c|c|c|c|c|c|c|}                                
\hline                                                          
\( X_0 \)  & \multicolumn{2}{c|}{0 } & \multicolumn{2}{c|}{0.500} & \multicolumn{2}{c|}{ 1  } \\       
\hline                                                          
Dane & X & Y & X & Y & X & Y \\                                
\hline                                                          
1  & -2.333 & -15.123 & -1.022 & -1.307 & 0.318 & 7.934 \\   
\hline                                                          
2  & -4.316 & 248.641 & -0.930 & -0.043 & -1.946 & -14.579 \\
\hline                                                          
3  & -3.591 & 71.539 & -0.927 &  0 & -0.256 & 6.026 \\   
\hline                                                          
4  & -3.144 & 17.787 & -0.927 & 0 & -1.493 & -8.632 \\  
\hline                                                          
5  & -2.934 & 2.878 &  &  & -0.946 & -0.255 \\     
\hline                                                          
6  & -2.884 & 0.138 &  &  & -0.927 & -0.002 \\     
\hline                                                          
7  & -2.881 & 0 &  &  & -0.927 & 0 \\     
\hline                                                          
8  & -2.881 & 0 &  &  &  &  \\       
\hline                                                          
9  &  &  &  &  &  &  \\        
\hline                                                          
\end{tabular}                                                   
\caption{Wyniki metody Newtona cz. 5}                           
\label{table:Wyniki metody Newtona cz. 5}                       
\end{table}                                                     
\begin{table}[p]                                                  
\centering                                                     
\begin{tabular}{|c|c|c|c|c|c|c|}                               
\hline                                                         
\( X_0 \) & \multicolumn{2}{c|}{1.500}  &\multicolumn{2}{c|}{ 2 }  & \multicolumn{2}{c|}{2.500 }\\      
\hline                                                         
Dane & X & Y & X & Y & X & Y \\                               
\hline                                                         
1  & 0.911 & 13.236 & 1.353 & 26.481 & 1.753 & 51.905 \\    
\hline                                                         
2  & 0.170 & 7.479 & 0.763 & 11.018 & 1.143 & 18.696 \\     
\hline                                                         
3  & -2.507 & -12.868 & -0.139 & 6.532 & 0.519 & 8.863 \\   
\hline                                                         
4  & -3.193 & 22.161 & -1.847 & -13.590 & -0.937 & -0.141 \\
\hline                                                         
5  & -2.952 & 3.945 & -0.629 & 3.393 & -0.927 & -0.001 \\   
\hline                                                         
6  & -2.886 & 0.245 & -0.988 & -0.827 & -0.927 &  0\\  
\hline                                                         
7  & -2.881 & 0.001 & -0.928 & -0.019 &  &  \\    
\hline                                                         
8  & -2.881 & 0 & -0.927 & 0 &  &  \\    
\hline                                                         
9  &  &  & -0.927 & 0 &  &  \\     
\hline                                                         
\end{tabular}                                                  
\caption{Wyniki metody Newtona cz. 6}                          
\label{table:Wyniki metody Newtona cz. 6}                      
\end{table}                                                    
\begin{table}[p]                                                                                                
\centering                                                                                                   
\hspace*{-2cm}\begin{tabular}{|c|c|c|c|c|c|c|}                                                                             
\hline                                                                                                       
\( X_0 \)  & \multicolumn{2}{c|}{-6 } &\multicolumn{2}{c|}{-5.500 }& \multicolumn{2}{c|}{-5  }\\                                                 
\hline                                                                                                       
Dane & X & Y & X & Y & X & Y \\                                                                             
\hline                                                                                                       
1  & -4.293+1.135i & 44.904-338.682i & -3.970+0.996i & 22.933-217.825i & -3.650+0.849i & 7.994-130.893i \\
\hline                                                                                                       
2  & -3.562+0.027i & 66.899-4.224i & -3.355+0.033i & 38.931-3.854i & -3.167+0.039i & 19.663-3.460i \\     
\hline                                                                                                       
3  & -2.790-0.228i & -6.902+9.575i & -2.819-0.022i & -3.100+1.006i & -2.872-0.004i & -0.466+0.228i \\     
\hline                                                                                                       
4  & -2.877-0.002i & -0.231+0.096i & -2.881+  & 0.003-0.004i & -2.881+  & 0  \\       
\hline                                                                                                       
5  & -2.881+  & 0  & -2.881-  & 0  & -2.881+  & 0  \\        
\hline                                                                                                       
6  &  &  &  &  &  &  \\                                                     
\hline                                                                                                       
7  &  &  &  &  &  &  \\                                                     
\hline                                                                                                       
8  &  &  &  &  &  &  \\                                                     
\hline                                                                                                       
9  &  &  &  &  &  &  \\                                                     
\hline                                                                                                       
\end{tabular}                                                                                                
\caption{Wyniki metody Mullera cz. 1}                                                                        
\label{table:Wyniki metody Mullera cz. 1}                                                                    
\end{table}                                                                                                  
\begin{table}[p]                                                                                             
\centering                                                                                                
\hspace*{-2cm}\begin{tabular}{|c|c|c|c|c|c|c|}                                                                          
\hline                                                                                                    
\( X_0 \)  & \multicolumn{2}{c|}{-4.500} & \multicolumn{2}{c|}{ -4 } &\multicolumn{2}{c|}{ -3.500 } \\                                              
\hline                                                                                                    
Dane & X & Y & X & Y & X & Y \\                                                                          
\hline                                                                                                    
1  & -3.335+0.688i & -1.361-71.042i & -3.027+0.495i & -6.408-32.008i & -2.731+0.182i & -8.234-6.658i \\
\hline                                                                                                    
2  & -3.010+0.043i & 7.535-2.891i & -2.905+0.032i & 1.217-1.735i & -2.878-0.002i & -0.195+0.109i \\    
\hline                                                                                                    
3  & -2.881-0.001i & -0.026+0.040i & -2.881  & 0.001  & -2.881  & 0 \\    
\hline                                                                                                    
4  & -2.881  & -  & -2.881  & 0  &  &  \\                  
\hline                                                                                                    
5  &  &  &  &  &  &  \\                                                  
\hline                                                                                                    
6  &  &  &  &  &  &  \\                                                  
\hline                                                                                                    
7  &  &  &  &  &  &  \\                                                  
\hline                                                                                                    
8  &  &  &  &  &  &  \\                                                  
\hline                                                                                                    
9  &  &  &  &  &  &  \\                                                  
\hline                                                                                                    
\end{tabular}                                                                                             
\caption{Wyniki metody Mullera cz. 2}                                                                     
\label{table:Wyniki metody Mullera cz. 2}                                                                 
\end{table}                                                                                               
\begin{table}[p]                                                
\centering                                                   
\begin{tabular}{|c|c|c|c|c|c|c|}                             
\hline                                                       
\( X_0 \)  & \multicolumn{2}{c|}{-3 } & \multicolumn{2}{c|}{-2.500  }& \multicolumn{2}{c|}{-2 } \\ 
\hline                                                       
Dane & X & Y & X & Y & X & Y \\                             
\hline                                                       
1  & -2.881 & -0.032 & -2.900 & 1.015 & -1.226 & -4.381 \\
\hline                                                       
2  & -2.881 & 0 & -2.881 & 0 & -0.934 & -0.105 \\
\hline                                                       
3  &  &  & -2.881 & 0 & -0.927 & 0 \\  
\hline                                                       
4  &  &  &  &  &  &  \\     
\hline                                                       
5  &  &  &  &  &  &  \\     
\hline                                                       
6  &  &  &  &  &  &  \\     
\hline                                                       
7  &  &  &  &  &  &  \\     
\hline                                                       
8  &  &  &  &  &  &  \\     
\hline                                                       
9  &  &  &  &  &  &  \\     
\hline                                                       
\end{tabular}                                                
\caption{Wyniki metody Mullera cz. 3}                        
\label{table:Wyniki metody Mullera cz. 3}                    
\end{table}                                                  
\begin{table}[p]                                                 
\centering                                                    
\begin{tabular}{|c|c|c|c|c|c|c|}                              
\hline                                                        
\( X_0 \)  & \multicolumn{2}{c|}{-1.500 } & \multicolumn{2}{c|}{-1 }& \multicolumn{2}{c|}{-0.500}\\  
\hline                                                        
Dane & X & Y & X & Y & X & Y \\                              
\hline                                                        
1  & -0.986 & -0.810 & -0.927 & -0.001 & -0.927 & -0.011 \\
\hline                                                        
2  & -0.927 & -0.001 & -0.927 & 0  & -0.927 & 0 \\
\hline                                                        
3  & -0.927 &  0 &  &  &  &  \\    
\hline                                                        
4  &  &  &  &  &  &  \\      
\hline                                                        
5  &  &  &  &  &  &  \\      
\hline                                                        
6  &  &  &  &  &  &  \\      
\hline                                                        
7  &  &  &  &  &  &  \\      
\hline                                                        
8  &  &  &  &  &  &  \\      
\hline                                                        
9  &  &  &  &  &  &  \\      
\hline                                                        
\end{tabular}                                                 
\caption{Wyniki metody Mullera cz. 4}                         
\label{table:Wyniki metody Mullera cz. 4}                     
\end{table}                                                   
\begin{table}[p]                                                                           
\centering                                                                              
\begin{tabular}{|c|c|c|c|c|c|c|}                                                        
\hline                                                                                  
\( X_0 \)  & \multicolumn{2}{c|}{0 } & \multicolumn{2}{c|}{0.500 }&  \multicolumn{2}{c|}{1 }\\                               
\hline                                                                                  
Dane & X & Y & X & Y & X & Y \\                                                        
\hline                                                                                  
1  & -1.266 & -5.004 & 0.162+0.957i & 8.452-2.850i & 0.560+0.637i & 5.328+1.923i \\  
\hline                                                                                  
2  & -0.938 & -0.157 & 0.637+0.895i & 1.052+0.467i & 0.647+0.953i & -0.104-0.381i \\ 
\hline                                                                                  
3  & -0.927 & 0 & 0.654+0.940i & -0.001-0.001i & 0.654+0.940i & 0  \\
\hline                                                                                  
4  &  &  & 0.654+0.940i & 0  & 0.654+0.940i & 0  \\   
\hline                                                                                  
5  &  &  &  &  &  &  \\                                
\hline                                                                                  
6  &  &  &  &  &  &  \\                                
\hline                                                                                  
7  &  &  &  &  &  &  \\                                
\hline                                                                                  
8  &  &  &  &  &  &  \\                                
\hline                                                                                  
9  &  &  &  &  &  &  \\                                
\hline                                                                                  
\end{tabular}                                                                           
\caption{Wyniki metody Mullera cz. 5}                                                   
\label{table:Wyniki metody Mullera cz. 5}                                               
\end{table}                                                                             
\begin{table}[p]                                                                                        


\centering                                                                                           


\hspace*{-2cm}\begin{tabular}{|c|c|c|c|c|c|c|}                                                                     
\hline                                                                                               
\( X_0 \)  & \multicolumn{2}{c|}{1.500} & \multicolumn{2}{c|}{2  } & \multicolumn{2}{c|}{2.500}\\                                            
\hline                                                                                               
Dane & X & Y & X & Y & X & Y \\                                                                     
\hline                                                                                               
1  & 0.892+0.588i & 5.893+7.559i & 1.217+0.633i & 7.791+18.269i & 1.542+0.717i & 11.503+37.932i \\
\hline                                                                                               
2  & 0.654+0.907i & 0.572+0.568i & 0.743+0.910i & -0.895+2.300i & 0.865+0.980i & -4.859+4.415i \\ 
\hline                                                                                               
3  & 0.654+0.940i &  & 0.654+0.940i & -0.010+0.004i & 0.652+0.945i & -0.066-0.122i \\ 
\hline                                                                                               
4  & 0.654+0.940i &   & 0.654+0.940i &  & 0.654+0.940i &  \\  
\hline                                                                                               
5  &  &  &  &  &  &  \\                                             
\hline                                                                                               
6  &  &  &  &  &  &  \\                                             
\hline                                                                                               
7  &  &  &  &  &  &  \\                                             
\hline                                                                                               
8  &  &  &  &  &  &  \\                                             
\hline                                                                                               
9  &  &  &  &  &  &  \\                                             
\hline                                                                                               
\end{tabular}                                                                                        


\caption{Wyniki metody Mullera cz. 6}                                                                
\label{table:Wyniki metody Mullera cz. 6}                                                            
\end{table}    
\end{itemize}
\item Wnioski : 
\begin{itemize}
\item Dla metod z pierwszego punktu : \\* Metoda bisekcji zbiega liniowo, a jej błąd jest związany jedynie z ilością iteracji.Z kolei metoda siecznych zbiega szybciej, jednak dla zbyt dużego przedziału [a,b] może nie osiągnąć wyniku, szczególnie jeżeli występuje punkt gdzie \( f'(x) = 0\).Dla podanaje funkcji obydwie metody dały satysfakcjonujące rezultaty
\item Dla metod z drugiego punktu : Obydwie metody okazały się dość skuteczne i dla podanych wartości startowych skutecznie znajdowały wartości miejsc zerowych. Dla wybranego przeze mnie przedziału Metoda MM2 była skuteczniejsza i znajdowała pierwiastki w mniejszej liczbie iteracji. Pozwoliła ona również znaleźć wartości pierwiastków urojonych. Jako że były one swoimi sprzężeniami, metoda znalazła tylko jeden z nich.
\end{itemize}
\end{enumerate}
\end{document}

