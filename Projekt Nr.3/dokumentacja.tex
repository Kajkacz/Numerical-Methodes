\documentclass[a4paper, 11pt]{article}
\author{Kajetan Kaczmarek}
\usepackage{amsmath}
\usepackage{graphicx}
\usepackage{listings}
\usepackage[T1]{fontenc}
\usepackage[utf8]{inputenc}
\usepackage[polish]{babel}
\usepackage{color} %red, green, blue, yellow, cyan, magenta, black, white
\definecolor{mygreen}{RGB}{28,172,0} % color values Red, Green, Blue
\definecolor{mylilas}{RGB}{170,55,241}


\lstset{language=Matlab,%
    %basicstyle=\color{red},
    basicstyle=\tiny,
    breaklines=true,%
    morekeywords={matlab2tikz},
    keywordstyle=\color{blue},%
    morekeywords=[2]{1}, keywordstyle=[2]{\color{black}},
    identifierstyle=\color{black},%
    stringstyle=\color{mylilas},
    commentstyle=\color{mygreen},%
    showstringspaces=false,%without this there will be a symbol in the places where there is a space
    numbers=left,%
    numberstyle={\tiny \color{black}},% size of the numbers
    numbersep=9pt, % this defines how far the numbers are from the text
    emph=[1]{for,end,break},emphstyle=[1]\color{red}, %some words to emphasise
    %emph=[2]{word1,word2}, emphstyle=[2]{style},    
}



\begin{document}
\title{Sprawozdanie MNUM \\* Projekt nr.3 \\* 
Zadanie 3.32 \\*}
\maketitle

\begin{enumerate}

\item Opis zastosowanych algorytmów : 
\begin{enumerate}
\item W pierwszym zadaniu, tj. znalezienie zer dla funkcji \[f(x) = 0.5xcos(x) - ln(x) \] użyłem dwóch metod. Założeniami dla obywdu metod była a) ciągłość, co jest oczywiste dla ww. funkcji, oraz b) różne znaki na krańcach przedziału, do czego odnoszę się poniżej. Zastosowane metody : \begin{itemize} \, \item Metoda bisekcji \\* \, W metodzie bisekcji na początek liczony jest punkt wypadający pomiędzy podanymi wejściowymi punktami, tj. \( x = \dfrac{a+b}{2}\) dla p. wejściowych a i b. Następnie sprawdzamy czy punkt ten jest naszym zerem z podaną dokładnością eps, czyli czy \( |f(x)| < eps \). Jeśli tak jest kończymy wykonywanie algorytmu, jeśli nie to sprawdzamy warunek \( f(a) f(b) < 0 \) i w zależności od wyniku zastępujemy lewy lub prawy koniec przedziału w którym szukamy wyliczonym x, tak, aby krańce przedziału nadal miały przeciwne znaki. Alternatywnym warunkiem wyjściowym z pętli jest \( |a-b|<eps \), czyli zbliżenie się do siebie punktów a i b tak, że dalsze obliczenia są niemożliwe.
\item Metoda Siecznych \\* Metoda siecznych jest podobna do metody bisekcji - szukamy zer przez zawężanie zakresu poszukiwań, warunki końcowe są więc takie same.Różny jest jednak algorytm wyznaczania kolejnego punktu : tutaj kolejne punkty wyznaczamy ze wzoru \[ x_i = x_{i-1} - f(x_{i-1})\dfrac{x_{i-1} - x_{i-2}}{f(x_{i-1}) - f(x_{i-2})} \]
Tak że łączenie kolejnych punktów daje nam sieczne naszej funkcji f(x) i przybliża jej zera.
\item Uwaga techniczna - założeniami obydwu metod są różne znaki funkcji na krańcach przedziału. Jako że warunek ten nie jest spełniony dla zadanego przedziału w mojej funkcji, a do tego ww. metody znajdują tylko jedno zero, podzieliłem zadany przedział [2,11] na dwa mniejsze , tj. [2,7] i [7,11] tak aby w każdym znajdowało się jedno zero, i aby spełniały one założenia metod.
\end{itemize}
\item W drugim zadaniu, tj. znalezienie zer wielomianu \( f(x) = 2x^4 + 5x^3 - 2x^2 +3x^3 +7 \) zastosowałem metody : 
\begin{itemize}
\item Metoda Newtona \, Metoda Newtona, zwana inaczej metodą stycznych , opiera się na wykorzystaniu iteracjnego wzoru \[ x_i = x_{i-1} - \dfrac{f(x)}{f(x)'} \]
\item Metoda Mullera MM2
\end{itemize}
\end{enumerate}

\item Kod moich programów 
\begin{itemize}
\item Funkcja main dla pierwszego zadania \\*
\lstinputlisting{P1_Main.m}
\item Pomocnicza funkcja licząca wartości naszej funkcji podanej dla zadania \\*
 \lstinputlisting{fzad.m}
 \item Funkcja licząca zera funkcji metodą bisekcji
 \lstinputlisting{bisect.m}
 \item Funkcja licząca zera funkcji metodą siecznych
 \lstinputlisting{secants.m}
  \item Funkcja main dla drugiego zadania
 \lstinputlisting{P2_Main.m}
   \item Funkcja licząca zera funkcji metodą Newtona
 \lstinputlisting{newton.m}
   \item Funkcja licząca zera funkcji metodą Mullera
 \lstinputlisting{muller.m}
\end{itemize}
\item
Wyniki : 
\begin{itemize}
\item  Dla zadania pierwszego obydwie metody zwróciły zbliżone wyniki, tj.\\*
\begin{center}

	\begin{tabular}{ l*{2}{c}r}
  \hline	
	Metoda & Zero nr.1 & Zero nr. 2 \\ \hline
    Metoda Bisekcji & 7.27703857421875	& 5.38775634765625	\\
    Metoda Siecznych & 7.27702154631274	& 5.38773923503257 \\
  \hline
  \end{tabular}

\end{center}
\includegraphics[width=\textwidth, height=\textheight, keepaspectratio]{Zad_1.jpg}
\end{itemize}
\item Wnioski : 
\end{enumerate}
\end{document}

