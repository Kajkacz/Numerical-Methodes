\documentclass[a4paper, 11pt]{article}
\author{Kajetan Kaczmarek}
\usepackage{amsmath}
\usepackage[T1]{fontenc}
\usepackage[utf8]{inputenc}
\usepackage[polish]{babel}

\begin{document}
\title{Sprawozdanie MNUM \\* Projekt nr.1 \\* 
Zadanie 1.1 \\*}
\maketitle

\begin{enumerate}

\item Opis zastosowanych algorytmów : 
\begin{enumerate}
\item Do sprawdzenia dokładności maszynowej komputera wykorzystałem prosty algorytm dzielący liczbę 1 przez 2 przy każdej kolejnej iteracji tak długo aż nie była ona równa 0.Dla mojego komputera otrzymałem wynik 1076
\item Przy drugim poleceniu wykorzystałem algorytm faktoryzacji Cholesky'ego Banachiewicza. Najpierw utworzyłem funkcje generateA, generateB oraz generateC, tworzące zbiory danych zgodnie z podanymi specyfikacjami.Następnie użyłem pomocniczych funkcji do faktoryzacji
 Cholesky'ego zgodnie z podanymi wzorami 
 \[ l_{ii} = \sqrt{a_{ii} - \sum_{ k = 1 }^{i-1}l^2_{ik} }  \]
  oraz 
 \[ l_{ji} =
  \dfrac
 { a_{ji} - \sum_{k=1}^{i-1}
  l_{jk} \cdot l_{ik}}{l_{ii} } \]
  Otrzymałem dzięki temu macierz która pomnożona przez swoją tranzpozycję odtworzy zadaną macierz A.
\item
\end{enumerate}
\end{enumerate}
\end{document}
